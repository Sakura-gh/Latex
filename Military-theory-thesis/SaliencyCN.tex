% XeLaTex
%\documentclass[review]{cvpr}
\documentclass[final]{cvpr}

\usepackage[UTF8]{ctex}
\usepackage{url}
%\usepackage{cvpr}
\usepackage{times}
\usepackage{epsfig}
\usepackage{graphicx}
\usepackage{amsmath}
\usepackage{amssymb}
\usepackage{subfigure}
\usepackage{overpic}

\usepackage{enumitem}
\setenumerate[1]{itemsep=0pt,partopsep=0pt,parsep=\parskip,topsep=5pt}
\setitemize[1]{itemsep=0pt,partopsep=0pt,parsep=\parskip,topsep=5pt}
\setdescription{itemsep=0pt,partopsep=0pt,parsep=\parskip,topsep=5pt}


\usepackage[pagebackref=true,breaklinks=true,colorlinks,bookmarks=false]{hyperref}


%\cvprfinalcopy % *** Uncomment this line for the final submission

\def\cvprPaperID{159} % *** Enter the CVPR Paper ID here
\def\confYear{CVPR 2020}
\def\httilde{\mbox{\tt\raisebox{-.5ex}{\symbol{126}}}}

\newcommand{\cmm}[1]{\textcolor[rgb]{0,0.6,0}{CMM: #1}}
\newcommand{\todo}[1]{{\textcolor{red}{\bf [#1]}}}
\newcommand{\alert}[1]{\textcolor[rgb]{.6,0,0}{#1}}

\newcommand{\IT}{IT\cite{98pami/Itti}}
\newcommand{\MZ}{MZ\cite{03ACMMM/Ma_Contrast-based}}
\newcommand{\GB}{GB\cite{conf/nips/HarelKP06}}
\newcommand{\SR}{SR\cite{07cvpr/hou_SpectralResidual}}
\newcommand{\FT}{FT\cite{09cvpr/Achanta_FTSaliency}}
\newcommand{\CA}{CA\cite{10cvpr/goferman_context}}
\newcommand{\LC}{LC\cite{06acmmm/ZhaiS_spatiotemporal}}
\newcommand{\AC}{AC\cite{08cvs/achanta_salient}}
\newcommand{\HC}{HC-maps }
\newcommand{\RC}{RC-maps }
\newcommand{\Lab}{$L^*a^*b^*$}
\newcommand{\mypara}[1]{\paragraph{#1.}}

\graphicspath{{figures/}}

% Pages are numbered in submission mode, and unnumbered in camera-ready
%\ifcvprfinal\pagestyle{empty}\fi
\setcounter{page}{1}

\begin{document}
% \begin{CJK*}{GBK}{song}

\renewcommand{\figref}[1]{图\ref{#1}}
\renewcommand{\tabref}[1]{表\ref{#1}}
\renewcommand{\equref}[1]{式\ref{#1}}
\renewcommand{\secref}[1]{第\ref{#1}节}
\def\abstract{\centerline{\large\bf 摘要} \vspace*{12pt} \it}

%%%%%%%%% TITLE

\title{南宋的灭亡——环境演变的必然\thanks{本文为军事理论课程论文
}}

\author{葛浩\quad \quad 3180103494\\
    Zhejiang University \quad 
    Computer Science and Technology\\
}

\maketitle
% \thispagestyle{empty}

%%%%%%%%% ABSTRACT
\begin{abstract}
南宋在科技、文化、工商业、造船业及航海技术等方面取得了巨大的成就,离不开外部环境的驱动。越是繁荣的迹象,在某种意义上越是体现了南宋的弱势。南宋想要借助这份“繁荣”获取一线生机,却在外部环境的掣肘中越陷越深。这一性质几乎决定了整个南宋未来命运的走向,某些政治方面的因素不过是加速或延缓了这段历史进程,当蒙古大军踏遍整个欧亚,平衡的环境终被打破,南宋早已注定的结局也必然会发生。\\
\newline
\centering
\textbf{关键字:}南宋,环境,依赖性,经济,军事科技与外交,航海与贸易,蒙古帝国
\end{abstract}




%%%%%%%%% BODY TEXT %%%%%%%%%%%%%%%%%%%%%%%%%%%%%%%%%%%%%%%%
\section{引言}

南宋时期民间贸易往来频繁、经济文化一片繁荣、科技发展也近乎是中国古代史的巅峰,如此大好形势,最终却落败于蒙古马蹄之下。这一切的原因,是统治者纸醉金迷、政策愚昧,是军队积贫积弱、战斗力不足,还是重文轻武,奸臣当道?精忠报国的岳飞含冤而死,过零丁洋的文天祥无奈叹息,似乎是这个政权内部出了大问题,诸多文章的论述也都是围绕着这些方面展开探讨的。

在本文看来,以上因素并不是宋朝灭亡的根本原因,这诸多浮于表面的问题仅仅是宋朝走向灭亡进程的催化剂。宋朝尤其是南宋时期的虚假繁荣,实际上是统治阶级乃至这个政权下所有阶层的一种被动性的应激反应,它们越是感到不安,就越是显现的繁华兴盛,对外部和平环境的“依赖性”也就越强。当这个“依赖性”的来源被蒙古强大施压下所造成的政治军事因素所打破时,繁荣的泡沫就会随之散去,灭亡也是必然。本文将以南宋为主要分析对象,重点以南宋的经济、军事科技与外交、航海与贸易、蒙古帝国的崛起这四个方面展开论述。


%%%%%%%%%%%%%%%%%%%%%%%%%%%%%%%%%%%%%%%%%%%%%%%%%%%%%%%%%%%%%%%%%%%%%%%%%%%%%%%%%
\section{宋朝的经济}

宋朝以“工商惠国”,是以商业为主的商品经济,商业税是国家税收的根本。江南地区手工业和商品经济发达,却严重缺乏重工业和货币铸造,因此南宋经济结构相对来说是比较畸形的。手工业和商业的发展主要是通过贸易和交换来实现的,是一种需求经济,这种经济模式最大的特点是特别依赖于和平的外部环境,一旦发生战争,就会立即崩溃。相对来说,重工业和货币铸造业是比较适合战时状态的,它能为消耗过度的国力提供一定程度上的支持。

南宋恰好处于战争频繁的年代。这个国家不得不面对战争,南宋在一百五十二年间共进行了4次北伐:岳飞北伐、隆兴北伐、开禧北伐、端平入洛。每一次北伐战争都会造成巨大的财政支出,南宋的财政赤字如影随形,帝国随时都有财政崩溃的可能。有粗略统计表明,南宋时期平均每年财政总收入9000万贯左右,而在南宋初期,光是军费开支就占有6000万贯\footnote{数据来源: 丛亦冰,王刃《南宋一百五十年》} \cite{南宋一百五十年}。

如果不考虑战争开销,南宋的财政收入确确实实体现了其繁荣昌盛的程度。要知道,明末时期国家财政收入也才是1000万两左右,一两白银兑一贯铜钱,也就是说此时明朝的财政收入只能勉强到达南宋的1/9,宋代当时的GDP几乎占到了当时全球的50\%,是封建王朝的巅峰\cite{你不可不知的经济学}。可遗憾的是,正如前面提到的那样,南宋的经济非常依赖于外部的和平环境,为了“和平”,必须先经历战争,于是南宋政府想到了一招,发行纸币“会子”。一定的纸币发行量确实可以缓解当时铸造资源短缺的现状,并促进经济发展,为军队提供更多的财政支持。但当时政府所采用的钱会中半纳税制,却加剧了钱荒\cite{宋代货币与货币流通研究}。当战争多次冲击的时候,纸币大量发行导致的通货膨胀使得发展到顶峰的经济崩溃了。下图展示了南宋期间铜钱贬值的概况\footnote{数据来源:漆侠《宋代经济史》} \cite{宋代经济史}:

\begin{figure}[h]
  	\begin{overpic}[width=\columnwidth]{image-20200514232426571.png}
    \end{overpic}
    \caption{南宋期间铜钱贬值趋势 
    }\label{fig:colorFre}
\end{figure}

是否可以放弃发行纸币这个方案呢?汪胜铎\cite{两宋货币史}从《宋会要辑稿》中考证出:"乾道(1165-1173)初年岁产铜26.3万余斤,而旧额则为705.7万余斤"。采冶业的衰落、铁铜矿资源被夺、铜钱外流等原因,决定了纸币的发行是必然。南宋别无选择。

这也验证了本文提出的观点,南宋经济的繁荣依赖于外界和平环境,当蒙古甚至是金国带来一定的战争压力时,政府只能选择加大纸币发行量以扩大财政,短期内经济虽然得到刺激,但战争爆发时,虚假的经济泡沫必然会被通货膨胀所打破。


%%%%%%%%%%%%%%%%%%%%%%%%%%%%%%%%%%%%%%%%%%%%%%%%%%%%%%%%%%
\section{宋朝的军事科技与外交}

封建帝朝一般都会采取愚民政策,但在宋朝尤其是南宋时期,科技反而得到了支持和快速发展。早在1126年,金人围攻汴京,李纲在守城时就用霹雳炮击退金兵,《靖康传信录》卷二说道:“夜发霹雳炮以击贼,军皆惊呼”;而1161年,南宋军队已经将霹雳炮装备在水师舰船上,用于海战\cite{霹雳炮} \cite{中国军事通览}。

\begin{figure}[h]
  	\begin{overpic}[width=\columnwidth]{300px-Songrivership3.jpg}
    \end{overpic}
    \caption{载有霹雳炮的水师船舰
    }\label{fig:colorFre}
\end{figure}

从火器的发展来看,南宋的科技已经达到了领先世界的水平,但这其实是出于无奈之举。如果没有战争,没有外界环境的压迫,火器也许还是娱乐性用品,进一步军事化应用也不会得到发展。发展火器,很大程度上是因为不得不发展,外界的压力已经迫在眉睫,北方的金国和蒙古国虎视眈眈,而自身军队实力却依旧不足。为了弥补没有大量铁器和战马的劣势,火药火炮才登上历史舞台,这也从侧面反映了南宋的军事形势已经危在旦夕。

外交方面也是同样的道理。北宋的外交依旧是以维持皇帝地位及疆土完整为主要目的,而到了南宋成立,新政权在风雨飘摇中,对付金、蒙却军事实力不足,皇帝的地位岌岌可危。同样为了获取暂时的和平环境,南宋选择了称臣纳贡的外交政策,维持宋金对峙的局面,才借机得以联合蒙古灭了金朝,然而蒙古却不像金朝一样接收求和\cite{宋代外交史},最终在外界高压下,委曲求全的外交政策失效了。除了对金、蒙两国的外交态度之外,为了保证贸易的畅通,南宋也选择放弃天朝上国的身份,发展了对等的外交,甚至和西南的小国南诏以及越南也能和平共存。这些都是由当时特定的环境决定的。

不少历史书和公开发表的论著中谈论到宋朝灭亡的原因时,都会提及“积贫积弱”、“重文轻武”、“兵权与调兵权的分离使得兵不识将,将不识兵”,但在当时宋蒙实力差距悬殊的大背景下,这些因素都不会对战争形势产生决定性的影响,不过是加速了战争的进程罢了。


%%%%%%%%%%%%%%%%%%%%%%%%%%%%%%%%%%%%%%%%%%%%%%%%%%%%%%%%%%%%%%%%%%%%%%
\section{宋朝的航海与贸易}

南宋发达的造船业和航海贸易很大程度上也是外界环境因素所致。日本历史学者宫崎正勝所著的《图解东亚史》\cite{图解东亚史}中这样描述:“受到北方的压迫下,南宋推展海洋贸易,利用戎克船建立了抵达南印度的商业圈”。书中提到南宋中期(公元十二世纪末)与北宋初期(公元十世纪末)相比,海上贸易的收益约增长了20倍,被称为亚洲第二次大航海时代。

\begin{figure}[h]
  	\begin{overpic}[width=\columnwidth]{image-20200515104437463.png}
    \end{overpic}
    \caption{第二次大航海时代相互竞争的贸易圈
    }\label{fig:colorFre}
\end{figure}

对比于明清时期的闭关锁国,南宋的海洋贸易象征着对外开放,似乎更接近现代化的社会,是中国历史社会发生了倒退吗?实际上,对于封建王朝来说,闭关锁国确实是巩固统治最好的手段,依赖外贸、面向海洋的发展倾向不完全是南宋朝廷的自觉选择\cite{战时状态},在很大程度上是不得不如此。原因有二:\newline
(1) 正如前文中提及的,南宋面临着因军费开支巨大而造成的财政困境,不得不发展海外贸易以拓展财政来源\newline
(2) 在外部环境上,对立政权阻隔了传统陆上丝绸之路,只能借助航海手段开辟海上陶瓷之路

总的来讲,南宋的航海贸易是一种“不情愿”的从封闭走向开放,是战时状态下战时经济的一种应激性反应,自元朝大一统至明清时期愈发严格的海禁政策也体现了在没有外部压迫力的情况下,封建政体对海外贸易的基本态度。

%%%%%%%%%%%%%%%%%%%%%%%%%%%%%%%%%%%%%%%%%%%%%%%%%%%%%%%%%%%%%%%%%%%%%%
\section{蒙古帝国的崛起}

南宋的衰败乃至灭亡,并不在于自身的弱小,而是作为对手的蒙古帝国过于强大了。很多文章认为宋朝的式微是国家积贫积弱,军队实力不济所致,但本文认为南宋的军事力量不仅不弱,反而很强大。可以比较这样一组数据\footnote{数据来源:Wikipedia} \cite{蒙古帝国}:\newline
(1) 蒙古征服西辽用了1年;\newline
(2) 蒙古征服花刺子模用了1年半;\newline
(3) 蒙古征服罗斯联盟(今俄罗斯)用了5年;\newline
(4) 蒙古征服波斯和阿拔斯王朝用了8年;\newline
(5) 蒙古征服西夏用了10年;\newline
(6) 蒙古征服金朝用了22年;\newline
(7) 而蒙古彻底打败人们印象中所谓文弱的南宋居然用了45年! 

《全球通史》\cite{全球通史}中这样描述,蒙古人迅猛地扫荡了中亚、中东和东欧,可是,在中国则陷入困境:同中国人的大规模战斗打打停停地持续了数十年。按照中国人的标准,宋朝是一个软弱无能的王朝,但对蒙古人来说,征服宋朝却比征服中东的穆斯林统治者要艰难得多。

从上面的数据对比和全球通史的描述中可以看出,南宋的军事实力并不弱,是蒙古帝国太强了,大规模的调兵遣将,以举国之兵进攻一国,灭亡大理后形成两面夹击,使用迂回包抄的战术,南宋的灭亡几乎是必然。因此即使没有重文轻武、没有兵将不识,没有“积贫积弱”,南宋真的有办法对付这支即将征服整个欧亚大陆的骁勇之师吗?答案是否定的。

这里不具体描述蒙古帝国崛起、征服欧亚的过程,但光是从下面这张疆域图中就可以感受到蒙古帝国的气势和魄力,它的崛起势不可挡,征服南宋只是这个崛起过程中的一步而已。

\begin{figure}[h]
  	\begin{overpic}[width=\columnwidth]{image-20200515115045065.png}
    \end{overpic}
    \caption{蒙古帝国的疆域图
    }\label{fig:colorFre}
\end{figure}

%%%%%%%%%%%%%%%%%%%%%%%%%%%%%%%%%%%%%%%%%%%%%%%%%

\section{总结与展望}\label{sec:Conclusion}

中国的地形和地理状况、战略物资的不平均分布,决定了大一统国家的政治形势。在宋代,北方拥有较为完整的工业体系,有铁矿、可养马;粮仓在蜀中和湖北;铜矿、银矿在云南;江南地区则拥有发达的手工业和商品经济。这也就导致了,任意一个分裂的地方政权,都无法摆脱对其他区域的物资依赖。

南宋所处江南地区,拥有粮仓和发达的手工业,这是商品经济繁荣的基础,沿海的便利更是提供了海上贸易的绝佳优势。但是,正如本文论述的那样,南宋缺乏重工业和货币铸造业,发展严重依赖于和平的外部环境,它在各个方面的繁荣,是在战时经济下的应激反应。纸币的发展是为了弥补铜银匮乏和军事开销,科技火器的发展是为了弥补军事实力的不足,平等的外交关系是基于不平等的政治局势,远洋贸易的发展是为了谋求更多的财政来源...

南宋曾表现出来很多特别现代化的特点,但这很大程度上是弥补自身劣势的无奈之举。当大一统帝国再次建立的时候,这些现代性就像退潮一般,消失的无影无踪了。

货币流通、科技发展、商业繁荣、外贸兴盛,南宋所表现出来的一切,都是在和平框架体系下所能做出的最大努力。南宋试图用这个框架带来的东西去打破框架本身,可看着眼前朝代繁华的假象又退却了,无论是统治者还是各个阶级都没能明白,自己已经对外部环境的依赖越陷越深。到了十三世纪中期,金国、西夏、南诏、朝鲜都已经是蒙古人的天下,日本封锁海面,防止蒙古的第二次入侵,强盛的阿拉伯此时只剩下阿拉伯半岛和埃及,自顾不暇。当时已知世界,所有曾经和南宋贸易的国家地区都被蒙古占领和在蒙古的威慑之下,南宋的财政出现极大的危机。南宋的统治者们终于意识到自己所依赖的东西已经被蒙古大军所击碎了,虚假的繁华如南柯一梦,南宋的灭亡,是遗憾也是必然。

南宋以贸易立国,最后败在贸易衰退。这个政权从成立的那一刻起,就已经将自己的命运捆绑在外部环境与大势之下了。它的性质决定了百年繁荣,也决定了必然衰亡的结局。

\section{附:论文说明}

严格说起来自己已经很久没有写过文史类的论文了,虽然高中学过一点历史,但对宋朝的印象却不是那么深刻了。起初写这篇文章的时候,查了不少资料都说是宋朝所采取的政策不对,批判政治、军事等的大有人在。我也查了不少资料,包括宋代的文化史、货币史以及贸易史等,看得越多就越发觉得自己认识得片面。遗憾的是这个学期学业负担着实有点重,这段时间通宵做各种大作业,因而也就没有耐心把各类史料仔细地阅读一遍。

但易中天先生写的《风流南宋》一书确实是饶有趣味,尽管跟我这篇文章的思想不同,却是启发我用另一种视角来看待南宋。既然没有办法在细节上去解释宋朝灭亡的原因,不妨跳脱出宋朝本身,从当时整个大局的角度去观察,竟然也能有一点收获。南宋所处的局势,几乎是一种处处被掣肘的情形,再联想为何能够在这种背景下获取经济、文化、航海贸易的繁荣,这篇文章的中心思想也就此产生了(写“虚假繁华”一词的时候总是会不自觉地联想到美国胡佛总统时期的经济危机)。尽管我的观点不一定正确,但也算是一种尝试吧。

写这篇文章还有一个尝试是,使用了LaTeX作为排版工具,因为使用word排版确实各种不便。虽然排版花了不是时间,但总体效果看起来还挺像印刷体的,感觉挺好。

\newpage

{\small
\bibliographystyle{unsrt}
\bibliography{Saliency}
}

% \end{CJK*}
\end{document}
